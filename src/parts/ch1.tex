\chapter{Introducción}
\section{Objetivo}

El objetivo de esta tesis es proveer de una interfaz simple y compacta para
especificar el comportamiento de modelos de simulación para la plataforma
\omnetpp{} mediante el uso del lenguaje Python.

\section{Hipótesis}

Es posible definir módulos simples de \omnetpp{} en scripts escritos en Python e
integrarlos de manera que el usuario experience mínima o nula complejidad en el
proceso.

\section{Metodología}

La metodología en la que se basa esta tesis es extender y embeber el intérprete
de Python para que pueda recibir y transmitir comportamiento desde y hacia C++.

\section{Impacto}

El resultado del trabajo (disponible en https://github.com/mmodenesi/omnetpy/)
resulta beneficioso para personas que buscan aprovechar la plataforma \omnetpp{}
para realizar simulaciones, pero para quienes la programación directa en C++
resulta un obstáculo. En particular, facilita la utilización de \omnetpp{} como
herramienta pedagógica en cursos de ciencias de computación donde los
estudiantes suelen contar con un dominio bastante bueno de Python, pero en
general desconocen C++.  El interés de este proyecto ha sido reconocido por los
desarrolladores de \omnetpp{} (https://omnetpp.org/download-items/omnetpy.html).

\section{Estructura}

El capítulo~\ref{sec:eda} de esta tesis presenta el estado del arte alrededor
de \omnetpp{} y la simulación de eventos discretos. Se realiza una comparación
entre C++ y Python. El capítulo~\ref{sec:met} presenta la metodología. El
capítulo~\ref{sec:des} discute el detalle del desarrollo. La herramienta
omnetpy se presenta en el capítulo~\ref{sec:omnetpy}. El capítulo~\ref{sec:ev}
evalúa cualitativa y cuantitativamente omnetpy. Las conclusiones se resumen en
el capítulo~\ref{sec:conc}, así como los desafíos abiertos.
