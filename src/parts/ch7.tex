\chapter{Conclusiones}\label{sec:conc}
Se presenta una extensión del simulador de eventos discreto \omnetpp{} que permite
escribir módulos simples en el lenguaje de programación Python. Esto habilita a
la escritura de modelos de simulación sin necesidad de saber programar en el
lenguaje oficial de \omnetpp{}, es decir, C++.

C++ es un lenguaje de programación que provee pocas abstracciones sobre el
hardware subyacente y que se compila para obtener un binario ejecutable
convierten en herramienta muy poderosa para escribir código de alta performance
(administración manual de la memoria, complejo sistema de tipos, posibilidad de
incluir código assembly), hacen de este un lenguaje de difícil dominio y con
tiempos de desarrollo relativamente largos, si los comparamos con otros
lenguajes como Python.

Python es un lenguaje de programación de mucho más alto nivel, donde los
detalles del hardware son alejados del usuario final. Un sistema de tipos más
flexible que requiere menos especificidad por parte del usuario, manejo
automático de la memoria, una sintaxis más cercana al lenguaje natural, hacen
de este un lenguaje muy idóneo para principiantes o para enseñanza en carreras
relacionadas a la computación. Los programas en Python no se compilan, sino que
son consumidos por otro programa (el intérprete de Python) para llevarlos a
instrucciones de la máquina virtual de Python y finalmente ejecutar tales
instrucciones una a una en el CPU. La implementación más popular de intérprete
está hecha en C y se conoce como cPython.

En general, los programas escritos en C++ son varias veces más rápidos que
programas con la misma funcionalidad escritos en Python, y los programas
escritos en Python son más fáciles de entender, modificar, extender. Es decir
que al movernos de C++ a Python, lo que se gana es mayor claridad en el código
y tiempos de desarrollo más cortos, mientras que si nos movemos en la otra
dirección, lo que se gana es mayor performance computacional.

La posibilidad de escribir simulaciones de \omnetpp{} usando Python en lugar de
C++ tiene varias ventajas asociadas:

\begin{itemize}
    \item menor barrera de entrada para usuarios nuevos

    \item mayor facilidad para incorporar \omnetpp{} en cursos de computación

    \item menor tiempo de prototipado de nuevos modelos

    \item inmediata disponibilidad de todos los recursos del ecosistema Python

    \item posibilidad de introducir cambios en el modelo sin pasos extra de
compilación
\end{itemize}

La extensión de la herramienta \omnetpp{} fue realizada sin necesidad de
modificar el código fuente original, lo cual facilita su aplicación en
distintas versiones de \omnetpp{}.

Las simulaciones ejecutadas con modelos hechos en Python arrojaron idénticos
resultados que las versiones originales en C++, aunque con penalidades de
performance variable (dependiendo de la complejidad del método \verb!handleMessage!).

\section{Desafíos abiertos}

Como se menciona en la sección~\ref{subsec:lim}, quedan algunos desafíos
abiertos para continuar con la tarea iniciada en este trabajo, los cuales se
detallan a continuación.

\subsection{Generación automática de librerías de extensión}

La extensión del intérprete con las librerías de \omnetpp{} se realizó
manualmente y exponiendo sólo aquellas partes que eran necesarias para portar
las simulaciones de ejemplo del proyecto original.

Sería interesante que este proceso fuera automatizado de forma que dada una
versión de \omnetpp{} se generaran los Python bindings de la totalidad de la
librería.

Las ventajas de esto serían:

\begin{enumerate}
    \item Que cualquier función, clase, método, macro, tipo definido en
    \omnetpp{} (en C++) fuera accesible y extensible desde Python

    \item Se podría contar fácilmente con versiones de omnetpy para cualquier
    versión de \omnetpp{} (recordar que el trabajo se basa en la versión 5.6.2 y se
    probó ligeramente sobre las versiones de perview de \omnetpp{} 6).

    \item En caso de emprender este esfuerzo, sería interesante dedicar algún
    tiempo a diseñar una infraestructura de testing. Por ej, realizar simulaciones
    en C++ y en Python y comparar que las salidas de sus ejecuciones sean
    idénticas, apuntando a que estas simulaciones utilicen el mayor porcentaje
    posible de las librerías.
\end{enumerate}

\subsection{Depurador interactivo en Python}

Actualmente, cuando un modelo hecho en Python no se comporta de la forma que se
espera y se dificulta encontrar el error, no es posible recurrir a establecer
un punto de depuración (debugging) y realizar una inspección interactiva y
pausada del estado de las variables. Esto es así porque el intérprete de Python
no es el programa principal, sino que está embebido en un programa en C++ (la
simulación).

Siendo la trazabilidad y la habilidad de depurar las simulaciones tan central
en el proyecto \omnetpp{}, y siendo que este trabajo apunta a que escribir
simulaciones sea más simple, perder esta funcionalidad parece un contrasentido
que merece la pena tratar de arreglar.

\subsection{Formas de distribución y disponibilización}

Durante el desarrollo de este trabajo se recurrió a congelar un ambiente
controlado dentro de una imagen Docker con versiones conocidas de todos los
elementos de software involucrados (compilador, \omnetpp{}, Python, pybind11,
etc). Esto por un lado permitió contar con un ambiente controlado y repetible
donde sabíamos que las cosas funcionaban y por otro tuvo la ventaja de hacer
que el código pudiera ser aprovechado desde cualquier computadora con muy pocos
pasos de preparación.

No obstante, a la larga esto va a convertirse en una limitación: los sistemas
operativos dejan de ser mantenidos, las librerías necesarias ya no se consiguen
o dejan de tener mantenimiento oficial, Python 3.6 en algún momento será
declarado oficialmente sin soporte y ya nadie escribirá librerías para
enriquecer su ecosistema. Sería interesante pensar en algún mecanismo más
general de compilación y distribución que pudiera fácilmente seguir el ritmo de
los nuevos lanzamientos (nuevas versiones de Python, de pybind11, etc).

En cuanto a la tecnología de virtualización elegida (Docker) probó ser muy
beneficiosa desde el punto de vista de la facilidad de instalación, pero sería
interesante documentar otras formas de usar omnetpy sin recurrir a ella.
